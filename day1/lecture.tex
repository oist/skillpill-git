%%%%%%%%%%%%%%%%%%%%%%%%%%%%%%%%%%%%%%%%%
% Beamer Presentation
% LaTeX Template
% Version 1.0 (10/11/12)
%
% This template has been downloaded from:
% http://www.LaTeXTemplates.com
%
% License:
% CC BY-NC-SA 3.0 (http://creativecommons.org/licenses/by-nc-sa/3.0/)
%
% Modified by Jeremie Gillet in November 2015 to make an OIST Skill Pill template
%
%%%%%%%%%%%%%%%%%%%%%%%%%%%%%%%%%%%%%%%%%

%----------------------------------------------------------------------------------------
%	PACKAGES AND THEMES
%----------------------------------------------------------------------------------------

\documentclass{beamer}

\mode<presentation> {

\usetheme{Madrid}

\definecolor{OISTcolor}{rgb}{0.65,0.16,0.16}
\usecolortheme[named=OISTcolor]{structure}

%\setbeamertemplate{footline} % To remove the footer line in all slides uncomment this line
%\setbeamertemplate{footline}[page number] % To replace the footer line in all slides with a simple slide count uncomment this line

\setbeamertemplate{navigation symbols}{} % To remove the navigation symbols from the bottom of all slides uncomment this line
}

\usepackage{graphicx} % Allows including images
\usepackage{booktabs} % Allows the use of \toprule, \midrule and \bottomrule in tables
\usepackage{textpos} % Use for positioning the Skill Pill logo

% For code displays
\usepackage{listings}
\usepackage{color}
\usepackage{amsmath}

\definecolor{dkgreen}{rgb}{0,0.6,0}
\definecolor{gray}{rgb}{0.5,0.5,0.5}
\definecolor{mauve}{rgb}{0.58,0,0.82}

\lstset{frame=tb,
  language=python,
  aboveskip=3mm,
  belowskip=3mm,
  showstringspaces=false,
  columns=flexible,
  basicstyle={\small\ttfamily},
  numbers=none,
  numberstyle=\tiny\color{gray},
  keywordstyle=\color{blue},
  commentstyle=\color{dkgreen},
  stringstyle=\color{mauve},
  breaklines=true,
  breakatwhitespace=true,
  tabsize=3
}



%----------------------------------------------------------------------------------------
%	TITLE PAGE
%----------------------------------------------------------------------------------------

\title[Skill Pill]{Skill Pill: Introduction to Git and Version Control} % The short title appears at the bottom of every slide, the full title is only on the title page
\subtitle{Lecture 1: Git ready!}

\author{\textbf{James Schloss}} % Your name
\institute[OIST] % Your institution as it will appear on the bottom of every slide, may be shorthand to save space
{
Okinawa Institute of Science and Technology \\ % Your institution for the title page
\textit{james.schloss@oist.jp} % Your email address
}
\date{June 1, 2016} % Date, can be changed to a custom date

\begin{document}

\setbeamertemplate{background}{\includegraphics[width=\paperwidth]{SPbackground.png}} % Adding the background logo

\begin{frame}
\vspace*{1.4cm}
\titlepage % Print the title page as the first slide
\end{frame}

\setbeamertemplate{background}{} % No background logo after title frame

\addtobeamertemplate{frametitle}{}{% Adding the Skill Pill logo on the title screen after title frame
\begin{textblock*}{100mm}(.8\textwidth,-1.25cm)
\includegraphics[height=2cm]{SPwhite.png}
\end{textblock*}}


\begin{frame}
\frametitle{Overview} % Table of contents slide, comment this block out to remove it
\tableofcontents % Throughout your presentation, if you choose to use \section{} and \subsection{} commands, these will automatically be printed on this slide as an overview of your presentation
\end{frame}

%----------------------------------------------------------------------------------------
%	PRESENTATION SLIDES
%----------------------------------------------------------------------------------------

%------------------------------------------------
\section{What is Version Control}
\section{Terminal Talk}

\section{Git basics}

\section{Working alone}

\begin{frame}[fragile]
\frametitle{Version Control}
\includegraphics[width=\textwidth]{Multiverse.jpg}
\begin{itemize}
\item Version control is a method that allows you to control different versions of things (Not necessarily universes).
\end{itemize}

\end{frame}

\begin{frame}[fragile]
\frametitle{Terminal Talk}
\begin{columns}

\column{0.7\textwidth}
\begin{itemize}
\item We had a terminal skill pill and I have included the cheatsheet from that. 
\item There is a GUI downloadable from GitHub called the \textbf{GitHub Desktop}. We will not be using this for religious reasons.
\item Everything we do will be usable on Sango.
\item We will be using a cheatsheet from here: \href{https://www.git-tower.com/learn/cheat-sheets/git}{\textbf{\color{blue}{https://www.git-tower.com/learn/cheat-sheets/git}}}
\end{itemize}

\column{0.3\textwidth}
\includegraphics[width=\textwidth]{terminal.png}
\end{columns}
\end{frame}

\begin{frame}[fragile]
\frametitle{Your first repository}


\begin{itemize}
\item A \textbf{repository} is a place to store code.
\begin{itemize}
\item There are many sites to host your repository on (github, bitbucket), including your own local machine.
\item All of the essential parts of your repository can be found in \textbf{.git} directory
\end{itemize}

\end{itemize}

\includegraphics[width=\textwidth]{Storage.jpg}

\end{frame}

\begin{frame}[fragile]
\frametitle{The local repo}
\begin{columns}
\column{0.7\textwidth}
Let's \textbf{git} started.
\begin{itemize}
\item To initialize a git repository, simply type \textbf{git init} in a directory (preferably empty for now)
\item This creates a folder \textbf{.git/}, where all your git information is held.
\item Git tracks \textbf{commits}. Check these commits with \textbf{git log}.
\item \textbf{git status} checks any changes since the last commit.
\item \textbf{git add} adds new files.
\item \textbf{git commit} commits anything git status shows in \color{green}{green}.
\end{itemize}
\column{0.3\textwidth}
\includegraphics[width=\textwidth]{git.jpg}
\end{columns}
\end{frame}

\begin{frame}[fragile]
\frametitle{Quick Exercise}
    \begin{block}{EXERCISE}
        \begin{enumerate}
        \item Open a terminal
        \item Create a new directory and run \textbf{git init}
        \item Create a file and run \textbf{git status}
        \item Use a combination of \textbf{git add} and \textbf{git commit} to add a new file to the git repository.
        \item Check the \textbf{git log}.
        \end{enumerate}
    \end{block}

\end{frame}

\begin{frame}[fragile]
\frametitle{Ignorance is bliss}

\begin{itemize}
\item Keep your repository clean! Do your best to commit as few images and data files as possible!
\item You can do this by ignoring certain file extensions in a \textbf{.gitignore} file.
\end{itemize}
\begin{columns}
\column{0.5\textwidth}
\begin{lstlisting}
# Example gitignore configuration
*.log
*.tar
*.gz
*.exe
*.dat
\end{lstlisting}
\column{0.5\textwidth}
\includegraphics[width=\textwidth]{IIB.jpg}
\end{columns}
\end{frame}

\begin{frame}[fragile]
\frametitle{Quick Exercise}
    \begin{block}{EXERCISE}
        \begin{enumerate}
        \item Touch multiple files with various extensions, one of which should be \textbf{.dat}.
        \item Ignore the \textbf{.dat} file, but commit all the others.
        \item Be sure to write a clear message describing what you did.
        \item Check the \textbf{git log}
        \end{enumerate}
    \end{block}

\end{frame}

\begin{frame}
\frametitle{\textbf{git} with it!}
\begin{columns}
\column{0.7\textwidth}
Now we move to the fun* stuff: working with \textbf{online repositories}.
\begin{itemize}
\item For this, we will be using \textbf{github}. 
\item To use an online repository, we need to synchronize our local machine with the master repository held elsewhere. This is done with the \textbf{clone} command.
\item From here, you can do the following:
\begin{itemize}
\item \textbf{git push} to push any changes you may have to the online repository.
\item \textbf{git pull} to take any changes from the 
\end{itemize}
\end{itemize}
\column{0.3\textwidth}
\includegraphics[width=\textwidth]{clone.jpg}
\end{columns}

*Here, the word \textit{fun} is subject to interpretation.
\end{frame}

\begin{frame}[fragile]
\frametitle{Quick Exercise}
    \begin{block}{EXERCISE}
        \begin{enumerate}
        \item Clone our skillpill repository (or a similar repository):
        \begin{lstlisting}
git clone git@github.com:oist/skillpill-git.git
        \end{lstlisting}
        or
        \begin{lstlisting}
git clone https://github.com/oist/skillpill-git.git
        \end{lstlisting}
        \item Working with a small group, make commits and push and pull stuff from that repo.
        \end{enumerate}
    \end{block}

\end{frame}

\begin{frame}
\frametitle{What it will feel like...}
\begin{columns}
\column{0.6\textwidth}
\begin{itemize}
\item git is not intuitive to start with, but it's the best way to work collaboratively with other people.
\item The more you use it, the more you will like it. Think Stockholm syndrome.
\end{itemize}
\column{0.4\textwidth}
\includegraphics[width=\textwidth]{gitxkcd.png}
\end{columns}
\end{frame}

\begin{frame}[fragile]
\frametitle{A fear of commitment}
\includegraphics[width=\textwidth]{no_commit.jpg}
\end{frame}

\begin{frame}[fragile]
\frametitle{Write clear commit messages!}
\includegraphics[width=\textwidth]{git_commit.png}
\end{frame}

\begin{frame}[fragile]
\frametitle{Checking out your versions}
\begin{columns}
\column{0.7\textwidth}
We now know how to work with both local and online repositories, but what about using different versions?
\begin{itemize}
\item \textbf{git checkout} allows you to view the repository at any old commit (found with \textbf{git log}).
\item You may also checkout specific files like so:
        \begin{lstlisting}
git checkout a1e8fb5 hello.py
        \end{lstlisting}
\item Note that the most recent commit is \textbf{HEAD} and the one just before that is \textbf{HEAD$\mathbf{\sim}$1}
\item This command will be used later, so keep it in mind! 
\end{itemize}
\column{0.3\textwidth}
\includegraphics[width=\textwidth]{check_him_out.jpg}
\end{columns}
\end{frame}

\begin{frame}[fragile]
\frametitle{Cleaning the stage}
\begin{columns}
\column{0.7\textwidth}
Finally, what is actually happening with your commits under the hood?
\begin{itemize}
\item Git has a staging area before commits that can be checked with \textbf{git status}. Anything in \textcolor{green}{green} is staged.
\item If you wish to unstage the commit, simply type \textbf{git reset}.
\item \textbf{git reset} will work for individual files and you may go back to any commit in the history.
        \begin{lstlisting}
git reset HEAD~1
        \end{lstlisting}
\item If you wish to undo a commit entirely, use the \textbf{git revert} command.
\item \textbf{git clean} will remove any untracked files.
\end{itemize}
\column{0.3\textwidth}
\includegraphics[width=\textwidth]{the-hook.jpg}
\end{columns}
\end{frame}

\begin{frame}[fragile]
\frametitle{Quick Exercise}
    \begin{block}{EXERCISE}
        \begin{enumerate}
        \item Stage a commit
        \item Unstage the commit
        \item Make a commit
        \item Undo the commit
        \end{enumerate}
    \end{block}

\end{frame}

\begin{frame}
\frametitle{Final Comments}
\begin{columns}
\column{0.7\textwidth}
\begin{itemize}
\item git is weird. It's not intuitive, but it's the best way to collaborate with people on open projects.
\item Whenever you are using git, think about other people and how they will perceive your comments. \textbf{Would you be able to understand your own cryptic commit messages?}
\item You will make mistakes. Don't worry about it. Your entire history is backed up already. Learn from your mistakes and don't make them again!
\item Listen to git. It's smarter than you.
\end{itemize}
\column{0.3\textwidth}
\includegraphics[width=\textwidth]{commit_book.jpg}
\end{columns}
\end{frame}

\end{document} 