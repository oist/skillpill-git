%%%%%%%%%%%%%%%%%%%%%%%%%%%%%%%%%%%%%%%%%
% Beamer Presentation
% LaTeX Template
% Version 1.0 (10/11/12)
%
% This template has been downloaded from:
% http://www.LaTeXTemplates.com
%
% License:
% CC BY-NC-SA 3.0 (http://creativecommons.org/licenses/by-nc-sa/3.0/)
%
% Modified by Jeremie Gillet in November 2015 to make an OIST Skill Pill template
%
%%%%%%%%%%%%%%%%%%%%%%%%%%%%%%%%%%%%%%%%%

%----------------------------------------------------------------------------------------
%	PACKAGES AND THEMES
%----------------------------------------------------------------------------------------

\documentclass{beamer}

\mode<presentation> {

\usetheme{Madrid}

\definecolor{OISTcolor}{rgb}{0.65,0.16,0.16}
\usecolortheme[named=OISTcolor]{structure}

%\setbeamertemplate{footline} % To remove the footer line in all slides uncomment this line
%\setbeamertemplate{footline}[page number] % To replace the footer line in all slides with a simple slide count uncomment this line

\setbeamertemplate{navigation symbols}{} % To remove the navigation symbols from the bottom of all slides uncomment this line
}

\usepackage{graphicx} % Allows including images
\usepackage{booktabs} % Allows the use of \toprule, \midrule and \bottomrule in tables
\usepackage{textpos} % Use for positioning the Skill Pill logo
\usepackage{tikz} % Allow modification of transparency for title slide image

% For code displays
\usepackage{listings}
\usepackage{color}
\usepackage{amsmath}
\usepackage{listing}

\definecolor{dkgreen}{rgb}{0,0.6,0}
\definecolor{gray}{rgb}{0.5,0.5,0.5}
\definecolor{mauve}{rgb}{0.58,0,0.82}

\lstset{frame=tb,
  language=python,
  aboveskip=3mm,
  belowskip=3mm,
  showstringspaces=false,
  columns=flexible,
  basicstyle={\small\ttfamily},
  numbers=none,
  numberstyle=\tiny\color{gray},
  keywordstyle=\color{blue},
  commentstyle=\color{dkgreen},
  stringstyle=\color{mauve},
  breaklines=true,
  breakatwhitespace=true,
  tabsize=3
}



%----------------------------------------------------------------------------------------
%	TITLE PAGE
%----------------------------------------------------------------------------------------

\title[Skill Pill]{Skill Pill: Introduction to Git and Version Control} % The short title appears at the bottom of every slide, the full title is only on the title page
\subtitle{Lecture 2: Git it on!}

\author[Christian Butcher] % Shows on every footer
	{\textbf{Valentin Churavy - 2016} \\ James Schloss - 2018 \\ Christian Butcher - 2019} % Title slide names
\institute[OIST] % Your institution as it will appear on the bottom of every slide, may be shorthand to save space
{
Okinawa Institute of Science and Technology \\ % Your institution for the title page
\textit{christian.butcher@oist.jp} % Your email address
}
\date{August 14, 2019} % Date, can be changed to a custom date

\begin{document}


\setbeamertemplate{background}{
\begin{tikzpicture}\node[opacity=0.6]{\includegraphics[width=\paperwidth]{SPbackground.png}}; % Adding the background logo
\end{tikzpicture}}

\begin{frame}
\vspace*{1.4cm}
\titlepage % Print the title page as the first slide
\end{frame}

\setbeamertemplate{background}{} % No background logo after title frame

\addtobeamertemplate{frametitle}{}{% Adding the Skill Pill logo on the title screen after title frame
\begin{textblock*}{100mm}(.8\textwidth,-1.25cm)
\includegraphics[height=2cm]{SPwhite.png}
\end{textblock*}}


\begin{frame}
\frametitle{Overview} % Table of contents slide, comment this block out to remove it
\tableofcontents % Throughout your presentation, if you choose to use \section{} and \subsection{} commands, these will automatically be printed on this slide as an overview of your presentation
\end{frame}

%----------------------------------------------------------------------------------------
%	PRESENTATION SLIDES
%----------------------------------------------------------------------------------------

%------------------------------------------------

\section{Recap}
\begin{frame}[fragile]{Recap}
  Yesterday we covered (don't forget to prefix with \textbf{git}):
  \begin{itemize}
    \item \textbf{clone} : Cloning a repository into a new directory
    \item \textbf{add} : Add file contents to the \textit{index}. This makes git track the file.
    \item \textbf{commit} : Record changes. Store the staged files as a new part of the history!
    \item \textbf{pull} : Updating from a \textit{remote}. Technicially a combination of \textbf{fetch} and \textbf{merge} by default.
    \item \textbf{push} : Update remote \textit{refs} and objects.
  \end{itemize}
  
  \textit{remote} : Another git repository. We used GitHub to provide this.
  
  \textit{index} : A single, large, binary file listing all files in the current branch with some extra information. Reflects the \lq\lq{}proposed next commit\rq\rq{}.
  
  \textit{refs} : Short for references. Can point to almost anything in git.
\end{frame}
\begin{frame}[fragile]{Interlude: .gitconfig}
  The file \emph{.gitconfig} can be used to set default options per user or per project. The user files is in \emph{\textasciitilde/.gitconfig}. Each option can also be set with git \textbf{config}.
  \begin{lstlisting}
  [user]
    email = v.churavy@gmail.com
    name  = Valentin Churavy
  [github]
    user = vchuravy
  [push]
    default = simple
  [rerere]
    enabled = true
  \end{lstlisting}
\end{frame}
\subsection{Remotes}
\begin{frame}[fragile]{Remotes}

Yesterday we introduced \textbf{GitHub}. GitHub is a service that offers you a solution to remotely store your repositories.
\begin{itemize}
    \item Git is \emph{Distributed} Version Control System (DVCS). Every copy of your repository, may it be remote or local, is independent of each other. There is no central master repository. 

    \item In order to synchronize these distributed copies we introduce the concept of a remote.
\pause
  \begin{block}{}
    git \textbf{remote}
  \end{block}
\pause
  \item There can be as many remotes as you want each with different names. When you clone a repository there will be one default remote called \textbf{origin}.
\end{itemize}
\end{frame}
\begin{frame}
  \begin{block}{Exercise}
    \begin{enumerate}
      \item Clone our repository from GitHub: \url{https://github.com/oist/skillpill-git}
      \item Fork it
      \item Add your fork of the repository as a remote to your local repository
      \item Push a change to your fork
      \item Open a pull request on GitHub against the original repository
    \end{enumerate}
  \end{block}
\end{frame}
\subsection{Branches}
\begin{frame}[fragile]{Branches}
  Since there git is decentralized there is no one state of the repository that is correct. To manage this complexity git has the notion of a branch. 
  \begin{itemize}
    \item Branches are parallel timelines and are lightweight, so branch often and branch early.
  \begin{block}{}
    \begin{itemize}
      \item git \textbf{branch}  Manages branches. 
      \item git \textbf{checkout} Switch between branches.
    \end{itemize}
  \end{block}
    \item Most repositories have a default branch called \textbf{master}. Branches are just names for points in the history.
    \item Once we start working with branches we have to ask ourselves how are we going to join them back up? We can do this by performing a merge.

    \item You can also associate a local branch with a remote branch by setting it as upstream. git push \textbf{-u}.
  \end{itemize}
\end{frame}

\begin{frame}
  \begin{block}{Exercise}
    \begin{enumerate}
      \item Create a new branch, based of master
      \item Add a few commits to your branch
      \item Change back onto master
      \item Check the contents of the file(s) you changed on your other branch whilst you\rq{}re on the master branch
    \end{enumerate}
  \end{block}
\end{frame}
\subsection{Merging}
\begin{frame}[fragile]{Merging}
Merging is the act of joining two branches together or to join two different branches. You will always merge \emph{from} a branch/remote into a branch.
  \begin{block}{}
    \begin{itemize}
      \item git \textbf{fetch}  Gets remote changes
      \item git \textbf{merge}  Merge changes (ff by default)
      \item git \textbf{add}    Resolve merge-conflict
    \end{itemize}

    Options for merge:
    \begin{description}
      \item[--no-commit] Performs the merge, but doesn't commit yet. Giving you the change to edit the merge commit.
      \item[--ff-only]   Aborts when we can't perform a fast-forward merge.
      \item[--abort]     Aborts current conflict-resolution and reset to previous state.
    \end{description}
  \end{block}

  You can visualize your history in many different ways, but the best way on the command line is.\\
  \begin{block}{}
  git log \textbf{--graph --decorate --oneline}
  \end{block}
\end{frame}

\begin{frame}
  \begin{block}{Exercise}
    \begin{enumerate}
      \item Merge your branch onto master
      \item Create a file with some content, branch and change the contents both in master and the new branch
      \item Merge the two branches and correct the merge conflicts
    \end{enumerate}
  \end{block}
\end{frame}

\subsection{Rebasing and Rewriting history}
\begin{frame}[fragile]{Rewriting History}
  Rebases are a way to create fast-forward merges, by altering \emph{history}. Each branch has a root commit from which it diverged from the original commit. By rebasing we change this root. This has a couple of side effects. 
  \begin{itemize}
    \item Linear commit history.
    \item No merge commits within a branch.
    \item commit-ids change.
  \end{itemize}

  \begin{block}{}
    \begin{itemize}
      \item git \textbf{pull --ff-only} Don't merge if there are conflict with the remote
      \item git \textbf{rebase} Perform a rebase
      \item git \textbf{rebase -i} Perform a interactive rebase
      \item git \textbf{push -f} Force push your changes
      \item git \textbf{pull --rebase} Perform a pull with a rebase
    \end{itemize}
  \end{block}
\end{frame}

\begin{frame}
  \begin{block}{Exercise}
    \begin{enumerate}
      \item create a branch, with some commits
      \item go back to master and do some additional work
      \item rebase your branch onto master
      \item merge your branch onto master
    \end{enumerate}
  \end{block}
\end{frame}

\begin{frame}[allowframebreaks,fragile]{Secrets!}
  \begin{block}{Autosquash}
    \begin{itemize}
      \item git config rebase.autosquash true
      \item git commit \textbf{--squash=}\emph{some-hash}
      \item git commit \textbf{--fixup=}\emph{some-hash}
     \end{itemize}
     Autosquash will reorder the commits appropriatly before you perform a git \textbf{rebase -i}.
  \end{block}
  \begin{block}{Blame}
    There is no such thing as \textit{good} code. If you are using git with people, chances are that something will break at some time and you need someone to blame. That's what git blame is for:
    \begin{lstlisting}
        git blame -L 1,3 file
    \end{lstlisting}
  \end{block}
  \framebreak
  \begin{block}{Stash}
    When you are moving between branches you sometines want to keep your non-commited changes associated with the branch you where doing them one.
    \begin{itemize}
      \item git \textbf{stash}
      \item git \textbf{stash pop}
    \end{itemize}
  \end{block}

  \begin{itemize}
    \item git commit \textbf{--amend} Amend the last commit.
    \item git add \textbf{-i} Interactive add
    \item git add \textbf{-p} Interactive add in patch mode.
    \item git \textbf{rm} Removes file.
    \item git \textbf{mv} Move file within repository
  \end{itemize}
\end{frame}

\section{Workflow}
\begin{frame}{Workflows}
  \centering
  Storytelling from the battlefields.
\end{frame}
\subsection{Documentation and other helpful material}
\begin{frame}{Documentation}
  \begin{itemize}
    \item The Git book: \url{https://git-scm.com/book}
    \item The Git help/man pages: git \textbf{help} or git \emph{command} --help
    \item Caching your password: \url{https://help.github.com/articles/caching-your-github-password-in-git/}
    \item SSH-keys: \url{https://help.github.com/categories/ssh/} 
    \item Workflow: \url{https://www.atlassian.com/git/tutorials/comparing-workflows/centralized-workflow}
    \item Understand .git \url{https://medium.freecodecamp.com/understanding-git-for-real-by-exploring-the-git-directory-1e079c15b807}
  \end{itemize}
\end{frame}

\end{document}
